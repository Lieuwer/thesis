\documentclass{scrartcl}
% \documentstyle{article}
\usepackage{comment}
\usepackage{amsmath}
\usepackage{caption}
\begin{comment}

\end{comment}
\usepackage{xcolor}
\newcommand\todo[1]{\textcolor{red}{#1}}

\providecommand{\comm}[1]{{\bf[ #1 ]}}
\providecommand{\commd}[1]{\comm{D: {#1}}}

\begin{document}
\title{Master Thesis Research Proposal: summary}
\subtitle{What can the parameters of IRT based learner models tell us?}
\author{Lieuwe Rekker}
\maketitle


\section{Introduction}
In this research proposal the research question, its subquestions, the methods to investigate them and the models investigated are introduced. In very concise form the research can be summarized as follows. The learner models investigated all contain a number of latent factors that are determined when the model is fitted to the data. This research hopes to find some clues as to under what circumstances the found values are no longer useful due to too much variation in the found value or too large a deviation from its true value.
\todo{Say what will be discussed in the different parts and that it is not very detailed}

\section{Intelligent Tutor System Data}
 Intelligent tutor systems (ITS) are computer programs in which students make exercises. The data from these systems that the models in this research is based on consists of lists of what student, answered what question and whether it was answered correctly. The list is in chronological order. Additionally there is some meta data that connects the questions to a small(er) number of skills. In this research data with one or more skills per question will be focused on, but a dataset with exactly one skill per question might be used as well. A general goal for many of the models trained on this kind of data is to maximize the accuracy of predicting whether a student will answer a particular question correctly. In the next section the goal of the models looked at is discussed and the goal of interest in this research is discussed further in the section thereafter.

 \section{Item Response Theory Based Models}
 The learner models used in this research all stem from item response theory (IRT). IRT uses the same form of data as what is available from ITSs. A difference is that this data comes from tests done in a short timespan rather than a prolonged duration of time. Due to the context IRT makes the assumption that learning does not occur (hence order of the data also doesn't matter). The incorporation of a learning rate is the extension made to IRT to arrive at the learner models described below. Additionally IRT is not so much interested in predicting how a student would perform on a question, but rather is interested in the latent factors, especially one that represents student skill. The gist of this research is to check if these latent factors are still accurate and stable once the model has been extended to incorporate learning and when it is applied to ITS data.
\todo{mention the models and give a reference}

\section{Research question and method}
The main research question is "When are parameter values found in fitting IRT based models consistent and accurate". This will be looked at from the perspective of generated data as well as data from ITSs. In the case of ITS data the true values of the parameters are unknown and will not be taken in consideration, unless they are available form previous research. Accuracy will thus only be looked at when considering generated data. As an alternative to accuracy while using ITS data, consistency will be looked at: similar to tests used in IRT the expectation is that parameter values should remain similar when taking different data from the same domain. The data will thus be split into parts and it is checked how much the parameter values vary between different runs.

The 'when' part of the question is further specified by the following subquestions. Underneath each question is given some rationale as to why this question is important and an indication how it will be answered in the research.

1. What is the influence of amount of data used for fitting? Amount of data is an obvious influence on the fitting of parameters. With a very small amount of data, the parameters are expected to completely overfit, while at some point increasing the data no longer improves accuracy or consistency. Controlling for this variable should help distinguishing whether a model is simply worse or needs more data than another model. 

2. What is the influence of using different models for fitting? The four different models from the previous section will all be used and compared with each other to see what role their differences play.

3. What is the influence of using data from different domains? In different ITSs, the underlying structure of the data may be completely different, due to factors such as what questions are asked, how the questions are link to skill and in what context the tutor is used. To have an exploratory look at this, a few different ITS data sets are used. In generating data, parameter values will be estimated from data to look into the effects of these domain specific parameter settings. Additionally models will also be trained on data generated by the other models. In generating every model is also adapted to a conjunctive model rather than an additive model \cite{skillcombi}. Inspecting the results from the generated data experiments might give some indication as to what assumptions of the models is violated in the ITS data.

4.what is the relationship between the accuracy of predictions made with the model, the likelihood of the data given the model and the accuracy/consistency of the fitted parameters? The kind of analyses performed in this research can be tedious. Therefor it is looked at if the simpler metrics of likelihood of the data and accuracy of prediction of the model on a testset (or a simple combination of the two) can be used as a solid indication that parameters are or aren't accurate and consistent.

5. What is the relationship between the accuracy and the consistency of the fitted parameters? This can only be done in the case of generated data, but serves an important purpose. It might indicate whether finding that parameters are consistent in the ITS data case is enough to conclude that the parameters are thus also accurate enough.

In the best case it is hoped that this research will tell when parameters found in fitting these models are reliable and can be used for various endeavours. Otherwise it is at least hoped that this research gives some tools to find out if the parameters in a particular case are reliable.
\bibliography{litlist}
\end{document}
